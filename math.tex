\documentclass{article}
\usepackage[utf8]{inputenc}

\title{NeurL}
\author{Benjamin Ciccarelli }
\date{February 2019}

\usepackage{natbib}
\usepackage{graphicx}

\begin{document}

\maketitle

\section{Mathematics Behind NeurL}

The output of a neuron can be described by
\[ \sigma (i_{1}*w_{1} + i_{2}*w_{2} ... i_{n}*w_{n} ) \]
or 
\[ O = \sigma (\sum_{x = 1}^{n} i_{x}*w_{x}) \]
where
\[ \sigma(x) = \frac{e^{x}}{1+e^{x}} \]
The partial derivative of the output with respect to a weight is
\[ \frac{\delta O_{n}}{\delta w_{n}} = \sigma'(\sum_{x = 1}^{n} i_{x}*w_{x}) * i_{n}  \]
or 
\[ \frac{\delta O_{n}}{\delta w_{n}} = \sigma'(\sigma^{-1}(O)) * i_{n}  \]
or 
\[ \frac{\delta O_{n}}{\delta w_{n}} = (O)*(1-O) * i_{n}  \]

\end{document}
